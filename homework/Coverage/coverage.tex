\documentclass[11pt]{article}
\usepackage{amsmath,amssymb,amsmath,amsthm,amsfonts}
\usepackage{latexsym,graphicx}
\usepackage{fullpage,color}
\usepackage{url,hyperref}
\usepackage{natbib}
\usepackage{graphicx,subfigure}
\usepackage{algorithm}
\usepackage{algorithmic}
\usepackage{listings}
\usepackage{xcolor}
\usepackage{color}

\numberwithin{equation}{section}

\pagestyle{plain}

\setlength{\oddsidemargin}{0in}
\setlength{\topmargin}{0in}
\setlength{\textwidth}{6.5in}
\setlength{\textheight}{8.5in}

\newtheorem{fact}{Fact}[section]
\newtheorem{question}{Question}[section]
\newtheorem{lemma}{Lemma}[section]
\newtheorem{theorem}[lemma]{Theorem}
\newtheorem{assumption}[lemma]{Assumption}
\newtheorem{corollary}[lemma]{Corollary}
\newtheorem{prop}[lemma]{Proposition}
\newtheorem{claim}{Claim}[section]
\newtheorem{remark}{Remark}[section]
\newtheorem{definition}{Definition}[section]
\newtheorem{prob}{Problem}[section]
\newtheorem{conjecture}{Conjecture}[section]
\newtheorem{property}{Property}[section]

\def\A{{\bf A}}
\def\a{{\bf a}}
\def\B{{\bf B}}
\def\bb{{\bf b}}
\def\C{{\bf C}}
\def\c{{\bf c}}
\def\D{{\bf D}}
\def\d{{\bf d}}
\def\E{{\bf E}}
\def\e{{\bf e}}
\def\F{{\bf F}}
\def\f{{\bf f}}
\def\g{{\bf g}}
\def\h{{\bf h}}
\def\G{{\bf G}}
\def\H{{\bf H}}
\def\I{{\bf I}}
\def\K{{\bf K}}
\def\k{{\bf k}}
\def\LL{{\bf L}}
\def\M{{\bf M}}
\def\m{{\bf m}}
\def\N{{\bf N}}
\def\n{{\bf n}}
\def\PP{{\bf P}}
\def\Q{{\bf Q}}
\def\q{{\bf q}}
\def\R{{\bf R}}
\def\rr{{\bf r}}
\def\S{{\bf S}}
\def\s{{\bf s}}
\def\T{{\bf T}}
\def\tt{{\bf t}}
\def\U{{\bf U}}
\def\u{{\bf u}}
\def\V{{\bf V}}
\def\v{{\bf v}}
\def\W{{\bf W}}
\def\w{{\bf w}}
\def\X{{\bf X}}
\def\x{{\bf x}}
\def\Y{{\bf Y}}
\def\y{{\bf y}}
\def\Z{{\bf Z}}
\def\z{{\bf z}}
\def\0{{\bf 0}}
\def\1{{\bf 1}}



\def\AM{{\mathcal A}}
\def\CM{{\mathcal C}}
\def\DM{{\mathcal D}}
\def\EM{{\mathcal E}}
\def\GM{{\mathcal G}}
\def\FM{{\mathcal F}}
\def\IM{{\mathcal I}}
\def\JM{{\mathcal J}}
\def\KM{{\mathcal K}}
\def\LM{{\mathcal L}}
\def\NM{{\mathcal N}}
\def\OM{{\mathcal O}}
\def\PM{{\mathcal P}}
\def\SM{{\mathcal S}}
\def\TM{{\mathcal T}}
\def\UM{{\mathcal U}}
\def\VM{{\mathcal V}}
\def\WM{{\mathcal W}}
\def\XM{{\mathcal X}}
\def\YM{{\mathcal Y}}
\def\RB{{\mathbb R}}
\def\RBmn{{\RB^{m\times n}}}
\def\EB{{\mathbb E}}
\def\PB{{\mathbb P}}

\def\TX{\tilde{\bf X}}
\def\TA{\tilde{\bf A}}
\def\tx{\tilde{\bf x}}
\def\ty{\tilde{\bf y}}
\def\TZ{\tilde{\bf Z}}
\def\tz{\tilde{\bf z}}
\def\hd{\hat{d}}
\def\HD{\hat{\bf D}}
\def\hx{\hat{\bf x}}
\def\nysA{{\tilde{\A}_c^{\textrm{nys}}}}

\def\alp{\mbox{\boldmath$\alpha$\unboldmath}}
\def\bet{\mbox{\boldmath$\beta$\unboldmath}}
\def\epsi{\mbox{\boldmath$\epsilon$\unboldmath}}
\def\etab{\mbox{\boldmath$\eta$\unboldmath}}
\def\ph{\mbox{\boldmath$\phi$\unboldmath}}
\def\pii{\mbox{\boldmath$\pi$\unboldmath}}
\def\Ph{\mbox{\boldmath$\Phi$\unboldmath}}
\def\Ps{\mbox{\boldmath$\Psi$\unboldmath}}
\def\ps{\mbox{\boldmath$\psi$\unboldmath}}
\def\tha{\mbox{\boldmath$\theta$\unboldmath}}
\def\Tha{\mbox{\boldmath$\Theta$\unboldmath}}
\def\muu{\mbox{\boldmath$\mu$\unboldmath}}
\def\Si{\mbox{\boldmath$\Sigma$\unboldmath}}
\def\si{\mbox{\boldmath$\sigma$\unboldmath}}
\def\Gam{\mbox{\boldmath$\Gamma$\unboldmath}}
\def\Lam{\mbox{\boldmath$\Lambda$\unboldmath}}
\def\De{\mbox{\boldmath$\Delta$\unboldmath}}
\def\Ome{\mbox{\boldmath$\Omega$\unboldmath}}
\def\Pii{\mbox{\boldmath$\Pi$\unboldmath}}
\def\varepsi{\mbox{\boldmath$\varepsilon$\unboldmath}}
\newcommand{\ti}[1]{\tilde{#1}}
\def\Ncal{\mathcal{N}}
\def\argmax{\mathop{\rm argmax}}
\def\argmin{\mathop{\rm argmin}}

\def\ALG{{\AM_{\textrm{col}}}}

\def\bias{\mathsf{bias}}
\def\var{\mathsf{var}}
\def\sgn{\mathsf{sgn}}
\def\tr{\mathsf{tr}}
\def\rk{\mathrm{rank}}
\def\nnz{\mathsf{nnz}}
\def\poly{\mathrm{poly}}
\def\diag{\mathsf{diag}}
\def\Diag{\mathsf{Diag}}
\def\const{\mathrm{Const}}
\def\st{\mathsf{s.t.}}
\def\vect{\mathsf{vec}}
\def\sech{\mathrm{sech}}

\newcommand{\red}[1]{{\color{red}#1}}



\def\argmax{\mathop{\rm argmax}}
\def\argmin{\mathop{\rm argmin}}

\newenvironment{note}[1]{\medskip\noindent \textbf{#1:}}%
        {\medskip}


\newcommand{\etal}{{\em et al.}\ }
\newcommand{\assign}{\leftarrow}
\newcommand{\eps}{\epsilon}

\newcommand{\opt}{\textrm{\sc OPT}}
\newcommand{\script}[1]{\mathcal{#1}}
\newcommand{\ceil}[1]{\lceil #1 \rceil}
\newcommand{\floor}[1]{\lfloor #1 \rfloor}



\lstset{ %
extendedchars=false,            % Shutdown no-ASCII compatible
language=Python,                % choose the language of the code
xleftmargin=1em,
xrightmargin=1em,
basicstyle=\footnotesize,    % the size of the fonts that are used for the code
tabsize=3,                            % sets default tabsize to 3 spaces
numbers=left,                   % where to put the line-numbers
numberstyle=\tiny,              % the size of the fonts that are used for the line-numbers
stepnumber=1,                   % the step between two line-numbers. If it's 1 each line
                                % will be numbered
numbersep=5pt,                  % how far the line-numbers are from the code   %
keywordstyle=\color[rgb]{0,0,1},                % keywords
commentstyle=\color[rgb]{0.133,0.545,0.133},    % comments
stringstyle=\color[rgb]{0.627,0.126,0.941},      % strings
backgroundcolor=\color{white}, % choose the background color. You must add \usepackage{color}
showspaces=false,               % show spaces adding particular underscores
showstringspaces=false,         % underline spaces within strings
showtabs=false,                 % show tabs within strings adding particular underscores
frame=single,                 % adds a frame around the code
%captionpos=b,                   % sets the caption-position to bottom
breaklines=true,                % sets automatic line breaking
breakatwhitespace=false,        % sets if automatic breaks should only happen at whitespace
%title=\lstname,                 % show the filename of files included with \lstinputlisting;
%                                % also try caption instead of title
mathescape=true,escapechar=?    % escape to latex with ?..?
escapeinside={\%*}{*)},         % if you want to add a comment within your code
%columns=fixed,                  % nice spacing
%morestring=[m]',                % strings
%morekeywords={%,...},%          % if you want to add more keywords to the set
%    break,case,catch,continue,elseif,else,end,for,function,global,%
%    if,otherwise,persistent,return,switch,try,while,...},%
}


\begin{document}

%\setlength{\fboxrule}{.5mm}\setlength{\fboxsep}{1.2mm}
%\newlength{\boxlength}\setlength{\boxlength}{\textwidth}
%\addtolength{\boxlength}{-4mm}


\title{CS583A: Coverage of the Final Exam}

\maketitle

\date{~~}

%\newpage
%
%\setcounter{tocdepth}{2}% Allow only \subsection in ToC
%
%\tableofcontents
%\newpage

%\section{Introduction}

%\paragraph{Policy:}
%Books and printed materials are allowed. Do not use electronic divice, including phone, laptop, and tablet.
%
%\paragraph{Hint:} (i) $\frac{\partial  e^a}{\partial a} = e^a$, (ii) $\frac{ \partial \log_e (a) }{\partial a  } = \frac{1}{a}$, (iii) $\frac{ \partial \frac{1}{a} }{\partial a  } = - \frac{1}{a^2}$, and (iv) $\frac{ \partial \cos (a) }{\partial a  } = - \sin (a)$.


\section{Vector and matrix basics.}

\begin{itemize}
	\item 
	Definitions of the $\ell_p$ vector norms for $p > 0$.
	\item
	Definitions and properties of the $\ell_p$-norm balls.
	\item
	Matrix trace and matrix norms.
	\item
	Singular value decomposition (SVD) and truncated SVD.
	\item
	The BLAS and LAPACK libraries for matrix computation. (What are level 1, 2, and 3 BLAS?)
\end{itemize}



\section{Scalar, vector, and matrix calculus.}

\begin{itemize}
	\item 
	Derivative of a scalar w.r.t.\ a vector.
	\item
	Derivative of a vector w.r.t.\ a scalar.
	\item
	Derivative of a vector w.r.t.\ a vector.
	\item
	Derivative of a scalar w.r.t.\ a matrix.
	\item
	Chain rule.
	\item
	Always check the shape of a derivative!
	\item
	Subgradient and subdifferential.
\end{itemize}




\section{Convex sets, convex functions, and convex optimization.}


\paragraph{Convex sets.}

\begin{itemize}
	\item 
	Definition of convex set.
	\item
	Typical examples of convex set and nonconvex set.
\end{itemize}




\paragraph{Convex functions.}

\begin{itemize}
	\item
	Definition of convex function.
	\item
	Definition of Hessian matrix. 
	\item
	Definition of positive semi-definite.
	\item
	For convex function, the Hessian matrix is everywhere positive semi-definite.
\end{itemize}





\paragraph{Convex and nonconvex optimization.}


\begin{itemize}
	\item
	Definition of convex optimization.
	\item
	Definitions of objective function, constrains, and feasible set.
	\item
	For convex optimization, local optimum is global optimum.
	\item
	For convex optimization, the first-order optimality condition ($0 \in \partial f(\w^\star)$) implies $w^\star$ is a global minimum.
	\item
	For nonconvex optimization, there are saddle points.
	\item
	Definition of saddle points.
	\item
	For high-dimensional nonconvex optimization,
	\#saddle points $\gg $ \#local minima $\gg$ \#global minima.
\end{itemize}


\section{Machine Learning Basics}

\paragraph{The four ML tasks.}

\begin{itemize}
	\item
	Definitions of regression, classification, clustering, and dimensionality reduction.
	\item
	Difference between regression and classification.
	\item
	Supervised learning and unsupervised learning.
\end{itemize}





\paragraph{Classification.}

\begin{itemize}
	\item
	Binary classification and multi-class classification methods, e.g., logistic regression, support vector machine (SVM), softmax classifier, and $k$-nearest neighbor (KNN).
	\item
	Linear classifiers include logistic regression, SVM, and softmax classifier.
	\item
	Nonlinear classifiers include KNN, kernel SVM, neural networks.
	\item
	What classification method is most suitable if \#classes is millions?
	\item
	Standard evaluation metrics: accuracy, classification error rate, top 1 classification error, top 5 classification error.
	\item
	Evaluation metrics for class-imbalanced problems: true positive, true negative, false positive, false negative, ROC curve, precision, and recall.
\end{itemize}







\paragraph{Clustering.}

\begin{itemize}
	\item
	Clustering tasks are unsupervised learning.
	\item
	The $k$-means clustering method (a combinatorial optimization model).
	\item
	Lloyd's algorithm for approximately solving the $k$-means model.
\end{itemize}





\paragraph{Dimensionality reduction.}

\begin{itemize}
	\item
	Unsupervised learning methods: PCA and autoencoder.
	\item
	Supervised learning method: linear discriminant analysis.
\end{itemize}




\paragraph{Model capacity, overfitting, and underfitting.}

\begin{itemize}
	\item
	What controls model capacity? E.g., degree of polynomial in polynomial regression, number of layers and width of layers in neural networks, etc.
	\item
	What are overfitting and underfitting?
	\item
	How to alleviate overfitting and underfitting?
	More training samples, regularization, and data augmentation.
\end{itemize}



\paragraph{Hyper-parameters and cross-validation.}

\begin{itemize}
	\item
	Examples of hyper-parameters:
	degree of polynomials, regularization parameter, neural network structure, optimization algorithms.
	(The model itself is a hyper-parameter.)
	\item
	Training set, validation set, and test set.
	\item
	Never use the test for hyper-parameter tuning.
\end{itemize}



\section{Convolutional Neural Networks}


\paragraph{Convolutional operations}

\begin{itemize}
	\item
	Definitions of patch, filter, matrix and tensor convolution, zero-padding, stride, etc.
	\item
	Calculate the output shape given the input shape, filter shape, stride, and zero-padding.
\end{itemize}


\paragraph{Convolutional neural networks}

\begin{itemize}
	\item
	Using Keras to implement convolutional layer, pooling layer, flatten layer, and dense layers.
	\item
	Being able to choose appropriate activation functions.
	\item
	Given the input shape, filter number, filter shape, stride, zero-padding, and pool size, infer the \textbf{output shape } and \textbf{number of parameters}.
	\item
	Tricks for allevating overfitting: regularization, data augmentation, and pretrain (with the bottom layers frozen).
	\item
	Other tricks for improving the test error: multi-task learning and ensemble method.
	\item
	What is dropout? What is the best place to insert a dropout layer?
	\item
	How to properly use pretrain? How to use fine-tuning?
	\item
	What is feature scaling? What is batch normalization? 
\end{itemize}




\section{Recurrent Neural Networks}

\begin{itemize}
	\item
	Definitions of token, sequence, encoding, word embedding, and vocabulary.
	\item
	Using Keras to implement SimpleRNN layer, LSTM layer, and Embedding layer.
	\item
	Given the sequence length, vocabulary size, and embedding dimension (shape of $x$), infer the \textbf{output shape} and \textbf{number of parameters} in an embedding layer.
	\item
	Given the sequence length, embedding dimension (shape of $x$), and state dimension (shape of $h$), infer the \textbf{output shape} and \textbf{number of parameters} in a SimpleRNN or LSTM layer.
	\item
	The advantage of LSTM over RNN.
	\item
	How to make LSTM less likely to forget?
	\item
	How to alleviate overfitting?
	\item
	Definitions of Seq2Seq models, attention, self-attention, multi-head attention, and Transformer model.
\end{itemize}


\end{document}
